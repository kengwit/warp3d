
\documentclass[11pt]{article}
\usepackage{geometry} 
\geometry{letterpaper}

%---------------------------------------------
\setlength{\textheight}{650pt}
\setlength{\textwidth}{450pt}
\setlength{\oddsidemargin}{14pt}
\setlength{\parskip}{1ex plus 0.5ex minus 0.2ex}
\setlength{\topmargin}{-0.5in}


%----------------------------------------
\usepackage{amsmath}
\usepackage{layout}
\usepackage{color}
\usepackage{array}
\usepackage{rotating}
\usepackage[us,12hr]{datetime}
\usepackage{lscape}  % for landscape pages in PDF


%----------------------------------------------
%\usepackage{fancyhdr} % \pagestyle{fancy}
%\setlength\headheight{15pt}
%\lhead{\small{\textit{Thermal Properties}}}
%\rhead{\small{R. H. Dodds, Jr. (rhdodds@gmail.com)}}
%\fancyfoot[L] {\small{\  Updated:  \today\ at \currenttime}}
%\fancyfoot[C] {\thepage}
%\fancyfoot[R] {\small{\textit{CTE Properties}}}

%---------------------------------------------------
\usepackage{graphicx}
\usepackage[labelformat=empty]{caption}
%\numberwithin{equation}{section}
\usepackage{bm}

%--------------------------------------------- 
%     --- make section headers in helvetica ---
%
\usepackage{sectsty} 
\usepackage{xspace}
\allsectionsfont{\sffamily} 
\sectionfont{\large}
\usepackage[small,compact]{titlesec} % reduce white space around sections
%---------------------------------------------->
%
%
%   which fonts system for text and equations. with all commented,
%   the default LaTex CM fonts are used
%
%
\frenchspacing
%\usepackage{pxfonts}  % Palatino text 
%\usepackage{mathpazo} % Palatino text
%\usepackage{txfonts}


%---------  local commands ---------------------

\newcommand{\ttt} {\texttt}  %typewriter text
\newcommand{\bmf } {\boldsymbol }  %bold math symbol
\newcommand{\bsf } [1]{\textrm{\textit{#1}}\xspace}
\newcommand{\ul} {\underline}
\newcommand{\hv} {\mathsf}   %helvetica text inside an equation
\newcommand{\eg}{\emph{e.g.},\xspace}
\newcommand{\ie}{\emph{i.e.},\xspace}
\newcommand{\ti}{\emph}
\newcommand{\vepsilon}{\varepsilon}
\newcommand{\etal}{\ti{et al.}\xspace}
\newcommand{\nid}{\noindent}
\newcommand{\vareps}{\varepsilon}

\newenvironment{offsetpar}[1]%
{\begin{list}{}%
         {\setlength{\leftmargin}{#1}}%
         \item[]%
}
{\end{list}}

%
%
%        optional definition for bullet lists which
%        reduces white space.
%
\newcommand{\squishlist}{
 \begin{list}{$\bullet$}
  { \setlength{\itemsep}{0pt}
     \setlength{\parsep}{3pt}
     \setlength{\topsep}{3pt}
     \setlength{\partopsep}{0pt}
     \setlength{\leftmargin}{1.5em}
     \setlength{\labelwidth}{1em}
     \setlength{\labelsep}{0.5em} } }

\newcommand{\squishlisttwo}{
 \begin{list}{$\bullet$}
  { \setlength{\itemsep}{0pt}
     \setlength{\parsep}{0pt}
    \setlength{\topsep}{0pt}
    \setlength{\partopsep}{0pt}
    \setlength{\leftmargin}{2em}
    \setlength{\labelwidth}{1.5em}
    \setlength{\labelsep}{0.5em} } }

\newcommand{\squishend}{
  \end{list}  }
%
\newcounter{Lcount}
\newcommand{\squishnum}{
\begin{list}{\arabic{Lcount}. }
{ \usecounter{Lcount}
\setlength{\itemsep}{0pt}
\setlength{\parsep}{3pt}
\setlength{\topsep}{3pt}
\setlength{\partopsep}{0pt}
\setlength{\leftmargin}{1in}
\setlength{\labelwidth}{1em}
\setlength{\labelsep}{0.5em} } }

\makeatletter
\renewcommand*\env@matrix[1][\arraystretch]{%
  \edef\arraystretch{#1}%
  \hskip -\arraycolsep
  \let\@ifnextchar\new@ifnextchar
  \array{*\c@MaxMatrixCols c}}
\makeatother

%-------------------------------------
%\newcounter{sectrefs}
%\setcounter{sectrefs}{0}
%\setcounter{section}{0}
%\setcounter{figure}{0}
%\renewcommand{\thefigure}{\thesection.\arabic{figure}}

%
%--------------------------------------
%
%
%
%              start document 
%              ==========
%
%
\begin{document}
\LARGE
\noindent \textbf{Periodic Boundary Conditions for RVEs}


\normalsize
\noindent \textbf{Python Code (rve\_mpc\_generator.py) to Construct Constraints
 for WARP3D\footnote{The generated constraint equations, absolute and multi-point,
 maybe readily converted into those required for Abaqus.}}\\
\noindent {(Rectangular Prism RVEs)}

\vspace{0.2 in}
\noindent Robert H. Dodds, Jr., PhD, NAE\\
\small
\noindent \rm{\it{rhdodds@gmail.com}}\\
{\small{\  Updated:  \today\ at \currenttime}}

\vspace{0.2 in}

\section{Purpose}
\label{app:pbc_script_logic}

The script generates a ready-to-use, WARP3D boundary-condition input file
 for a 3D representative volume element (RVE) using periodic 
 displacement boundary conditions (PBCs).
 The 3D mesh must be a rectangular prism constructed of hex or tet elements and
 satisfy the node-pairing requirement. Pairs 
of nodes are located on opposite faces of the RVE whose
in-plane coordinates are identical. They represent the same physical point in the periodic 
material and are coupled by the periodic displacement constraints. The codes NEPER and
GMSH have capabilities to generate meshes with node pairing.
The generated output file is written in WARP3D format:
\begin{itemize}
  \item a \texttt{constraints} block (absolute displacement constraints), followed by
  \item a \texttt{multipoint} block (multi-point constraints, MPCs).
\end{itemize}
No manual editing of the generated constraints/MPC file should be 
required for a correctly defined periodic RVE mesh. 
%
\begin{figure}[h]
\begin{center}
\includegraphics[trim=0.0in 5.5in 0.4in 0.0in, clip=true,scale=0.5,angle=0]{Fig_1.pdf} 
\caption{{\small Fig. 1. (a) Labelling of vertexes, edges and faces 
for rectangular prism RVE. (b) Example periodic displacements}.
\label{fig:macroscale}}
%
\end{center}
\end{figure}


\nid Figure 1 illustrates a rectangular prism RVE with some notation used in this
document.

The periodic loadings are specified by
\begin{subequations}\label{EE:e}
\begin{align}
u_1^{j+} - u_1^{j-}
&= \bar\varepsilon_{11}\,\Delta x_1^{\,j}
 + \bar\varepsilon_{12}\,\Delta x_2^{\,j}
 + \bar\varepsilon_{13}\,\Delta x_3^{\,j},\\
u_2^{j+} - u_2^{j-}
&= \bar\varepsilon_{21}\,\Delta x_1^{\,j}
 + \bar\varepsilon_{22}\,\Delta x_2^{\,j}
 + \bar\varepsilon_{23}\,\Delta x_3^{\,j},\\
u_3^{j+} - u_3^{j-}
&= \bar\varepsilon_{31}\,\Delta x_1^{\,j}
 + \bar\varepsilon_{32}\,\Delta x_2^{\,j}
 + \bar\varepsilon_{33}\,\Delta x_3^{\,j}\ .
\end{align}
\end{subequations}

\nid where $\bar\varepsilon_{ik}$ denotes the known, average strain tensor 
at a material point to be imposed on the RVE. This general form is
used to support algorithmic enforcement of periodicity and does 
not imply that the imposed field corresponds to a physically 
admissible macroscopic strain in the classical sense. In the present
implementation, $\bar\varepsilon_{ik}$ is not required to be symmetric. 
Example:  $\bar\varepsilon_{12}$ may be different than  $\bar\varepsilon_{21}$.
There exists no strict requirement that $\bar\varepsilon_{ik}$ actually
define a real, macroscale strain tensor. Here it is an often employed framework
to impose periodic loadings on the RVE.

Table 1 lists the symbolic MPC equations prior to elimination of 
zero terms, absolute constraints, and dummy-node substitutions. 

\section{Input data required by the script}
The input is a text file describing the RVE mesh and loading. 
Comment lines (or inline comments) may be included using the character \texttt{\#}.

\nid Required items include:
\begin{itemize}
\item number of nodes, excluding additional dummy-nodes required to enforce the PBCs. 
Element data may be present but is not used by the script,
\item declared dimensions $L_x$, $L_y$, $L_z$ of the rectangular prism 
(used for reporting and consistency checks),
\item node numbers for the 8 vertex nodes $A$--$H$,
\item the imposed $3\times 3$ macroscopic strain tensor 
$\bar{\varepsilon}_{ij}$ (row-wise input). Does not need to be symmetric,
\item an optional list of absolute constraints (DOFs fixed to zero). Usually to prevent 
rigid-body motion,
\item a \texttt{DUMMY\_EPS\_MAP} block defining the dummy-node method for 
each strain component used, and
\item the $X,Y,Z$ coordinates of all nodes. Nodes must be numbered sequentially. 
Coordinate values are truncated 
at 7 significant figures after reading. The code expects the dummy-node pairs
to be defined after all nodes of the RVE. Values for dummy nodes are ignored when included.
\end{itemize}

\begin{landscape}

\begin{center}
  \small
  {%
    \setlength{\extrarowheight}{6pt}
   \vspace*{0.5cm}
    %========================================
    % TABLE BLOCK (no float)
    %========================================
    \begin{tabular}{|>{\centering\arraybackslash}p{0.32\textwidth}|
                     >{\centering\arraybackslash}p{0.38\textwidth}|
                     >{\centering\arraybackslash}p{0.36\textwidth}|}
      \hline
      \textbf{Face Nodes} & \textbf{Vertex Nodes} & \textbf{Edge Nodes} \\
      \hline

      % Row 1
      $u_i^{\mathrm{FGCB}} - u_i^{\mathrm{EHDA}}
        - L_x\,\bar\varepsilon_{i1} = 0$
      &
      $u_i^{\mathrm{G}} - u_i^{\mathrm{A}}
        - L_x\,\bar\varepsilon_{i1}
        - L_y\,\bar\varepsilon_{i2}
        - L_z\,\bar\varepsilon_{i3} = 0$
      &
      \text{$\mathrm{FG} \leftrightarrow \mathrm{AD}$:}\par
      $u_i^{\mathrm{FG}} - u_i^{\mathrm{AD}}
        - L_x\,\bar\varepsilon_{i1}
        - L_y\,\bar\varepsilon_{i2} = 0$
      \\[14pt]
      \hline

      % Row 2
      $u_i^{\mathrm{FEHG}} - u_i^{\mathrm{BADC}}
         - L_y\,\bar\varepsilon_{i2} = 0$
      &
      $u_i^{\mathrm{F}} - u_i^{\mathrm{D}}
         - L_x\,\bar\varepsilon_{i1}
         - L_y\,\bar\varepsilon_{i2}
         + L_z\,\bar\varepsilon_{i3} = 0$
      &
      \text{$\mathrm{HG} \leftrightarrow \mathrm{AB}$:}\par
      $u_i^{\mathrm{HG}} - u_i^{\mathrm{AB}}
         - L_y\,\bar\varepsilon_{i2}
         - L_z\,\bar\varepsilon_{i3} = 0$
      \\[14pt]
      \hline

      % Row 3
      $u_i^{\mathrm{GHDC}} - u_i^{\mathrm{FEAB}}
         - L_z\,\bar\varepsilon_{i3} = 0$
      &
      $u_i^{\mathrm{H}} - u_i^{\mathrm{B}}
         + L_x\,\bar\varepsilon_{i1}
         - L_y\,\bar\varepsilon_{i2}
         - L_z\,\bar\varepsilon_{i3} = 0$
      &
      \text{$\mathrm{GC} \leftrightarrow \mathrm{EA}$:}\par
      $u_i^{\mathrm{GC}} - u_i^{\mathrm{EA}}
         - L_x\,\bar\varepsilon_{i1}
         - L_z\,\bar\varepsilon_{i3} = 0$
      \\[14pt]
      \hline

      % Row 4
      &
      $u_i^{\mathrm{C}} - u_i^{\mathrm{E}}
         - L_x\,\bar\varepsilon_{i1}
         + L_y\,\bar\varepsilon_{i2}
         - L_z\,\bar\varepsilon_{i3} = 0$
      &
      \text{$\mathrm{BC} \leftrightarrow \mathrm{EH}$:}\par
      $u_i^{\mathrm{BC}} - u_i^{\mathrm{EH}}
         - L_x\,\bar\varepsilon_{i1}
         + L_y\,\bar\varepsilon_{i2} = 0$
      \\[14pt]
      \hline

      % Row 5
      &
      &
      \text{$\mathrm{FE} \leftrightarrow \mathrm{CD}$:}\par
      $u_i^{\mathrm{FE}} - u_i^{\mathrm{CD}}
         - L_y\,\bar\varepsilon_{i2}
         + L_z\,\bar\varepsilon_{i3} = 0$
      \\[14pt]
      \hline

      % Row 6
      &
      &
      \text{$\mathrm{FB} \leftrightarrow \mathrm{HD}$:}\par
      $u_i^{\mathrm{FB}} - u_i^{\mathrm{HD}}
         - L_x\,\bar\varepsilon_{i1}
         + L_z\,\bar\varepsilon_{i3} = 0$
      \\[14pt]
      \hline

    \end{tabular}

    % Table caption (non-floating)
    \captionof{table}{{\small Table 1. MPC equations for faces, vertices, and edges in
    periodic boundary conditions. $i=1,2,3$}}
    \label{tab:node_equations}

    %========================================
    % Small vertical gap before figure
    %========================================
    \vspace*{0.1cm}

    %========================================
    % FIGURE BLOCK (no float)
    %========================================
    % NOTE: In landscape coordinates this is "below" the table.
    % When viewed in portrait, it appears "to the right" of the table.
    \includegraphics[trim=0.0in 6.0in 3.4in 0.0in, clip=true,scale=0.6,angle=0]{Fig_1.pdf}

    % (Caption omitted per your preference; you could add \captionof{figure} here if desired.)

  }% end group with extrarowheight
\end{center}

\end{landscape}


\section{Origin-independence via automatic detection of RVE bounds}
The script does \emph{not} require that the RVE occupy 
$x\in[0,L_x]$, $y\in[0,L_y]$, $z\in[0,L_z]$.
Instead, the script detects the geometric bounds directly from the node coordinates:
\begin{align}
x_{\min} &= \min(x), & x_{\max} &= \max(x), \\
y_{\min} &= \min(y), & y_{\max} &= \max(y), \\
z_{\min} &= \min(z), & z_{\max} &= \max(z),
\end{align}
and computes the detected dimensions:
\begin{align}
L_x^{\mathrm{det}} &= x_{\max}-x_{\min}, &
L_y^{\mathrm{det}} &= y_{\max}-y_{\min}, &
L_z^{\mathrm{det}} &= z_{\max}-z_{\min}.
\end{align}
The script reports both the declared dimensions $(L_x,L_y,L_z)$ and the detected dimensions
$(L_x^{\mathrm{det}},L_y^{\mathrm{det}},L_z^{\mathrm{det}})$ and issues 
a warning if they differ beyond tolerance.


\section{Boundary node classification}

Each RVE node is classified according to how many boundary planes it lies on:
\begin{itemize}
\item \ul{vertex node}: on the intersection of three boundary planes. 
The 8 nodes are labelled $A$--$H$.
\item \ul{edge node}: on the intersection of two boundary planes, excluding vertex nodes
\item \ul{face node}: on exactly one boundary plane, excluding vertex and edge nodes
\item \ul{interior node}: on no boundary plane,
\end{itemize}
Boundary planes are detected using the automatically determined bounds:
\begin{itemize}
\item $X^{-}$ plane at $x=x_{\min}$, \quad $X^{+}$ plane at $x=x_{\max}$,
\item $Y^{-}$ plane at $y=y_{\min}$, \quad $Y^{+}$ plane at $y=y_{\max}$,
\item $Z^{-}$ plane at $z=z_{\min}$, \quad $Z^{+}$ plane at $z=z_{\max}$.
\end{itemize}
A coordinate tolerance (relative to the largest RVE dimension) 
is used when assigning nodes to planes.

\section{Construction of periodic node pairings}
\subsection{Face-node pairings}
Nodes on opposite faces are paired by matching the two transverse coordinates (within tolerance):
\begin{itemize}
\item $X^{-}\leftrightarrow X^{+}$: match $(y,z)$,
\item $Y^{-}\leftrightarrow Y^{+}$: match $(x,z)$,
\item $Z^{-}\leftrightarrow Z^{+}$: match $(x,y)$.
\end{itemize}
For each paired node $(\,-,+\,)$, the script computes the geometric offset vector
\begin{equation}
\Delta \mathbf{x} = \mathbf{x}^{+}-\mathbf{x}^{-},
\end{equation}
which is translation-invariant and therefore independent of the global origin.



\subsection{Edge-node pairings}

Edge nodes are identified by the set of two boundary planes they lie on, \eg\ $\{X^{-},Y^{-}\}$.
Each family of opposite edges is paired using the remaining ``free'' coordinate as the edge parameter:
\begin{itemize}
\item $\{X^{-},Y^{-}\}\leftrightarrow\{X^{+},Y^{+}\}$, parameter axis $z$,
\item $\{X^{-},Y^{+}\}\leftrightarrow\{X^{+},Y^{-}\}$, parameter axis $z$,
\item $\{X^{-},Z^{-}\}\leftrightarrow\{X^{+},Z^{+}\}$, parameter axis $y$,
\item $\{X^{-},Z^{+}\}\leftrightarrow\{X^{+},Z^{-}\}$, parameter axis $y$,
\item $\{Y^{-},Z^{-}\}\leftrightarrow\{Y^{+},Z^{+}\}$, parameter axis $x$,
\item $\{Y^{-},Z^{+}\}\leftrightarrow\{Y^{+},Z^{-}\}$, parameter axis $x$.
\end{itemize}
For each paired edge-node pair, $\Delta\mathbf{x}$ is computed from coordinates as for face nodes.

\subsection{Vertex-node pairings}
The user supplies the node numbers for the eight vertices $A$--$H$.
The script forms the four body-diagonal vertex pairings and computes
$\Delta\mathbf{x}$ directly from the vertex coordinates; the origin 
location does not matter.


\section{Dummy node method for WARP3D MPC implementation}
WARP3D supports only multi-point equations (MPCs) that are homogeneous, \eg
\begin{equation}
\hv{27\ 1.0\ w\ -\ 20\ 1.0\ w\ - 49\ 3.5\ w = 0}
\end{equation}
To make an equivalent, non-homogenous equation required to impose non-zero terms of 
$\bar \varepsilon_{ij}$, WARP3D uses the so-called dummy-node and rigid-link
element method. The non-homogeneous terms 
are shifted to displacements applied at auxiliary (dummy) nodes, so that all MPCs 
become homogeneous. This greatly simplifies assembly of the equations and is compatible 
with the WARP3D existing constraint infrastructure. A similar procedure is often used to
enforce PBCs in Abaqus. 

%
\begin{figure}[h]
\begin{center}
\includegraphics[trim=0.0in 6.0in 1.0in 0.0in, clip=true,scale=0.6,angle=0]{Fig_2.pdf} 
\caption{{\small Fig. 2. Example demonstrating use of dummy nodes and rigid-link element
to enable definition of homogeneous multi-point constraints. Local nodes 1, 2 of the
link element are often coincident.}
\label{fig:macroscale}}
%
\end{center} 
\end{figure}
Figure 2 shows the setup for a simplified application of the dummy node-rigid link element
method. The $\hv{link2}$ element is connected to \rm{\it{{dummy}}} nodes 49 and 50. For
convenience these nodes are often defined well outside the RVE -- their actual (global)
position is immaterial. Spring stiffness
values for the element in global $(x,y,z)$ directions are set to a sufficiently large value.
Absolute constraints are imposed on node 50, \eg
\begin{equation}
\hv{50\ \ u\ = 0\  \ v  = 0\  \ w = 0.0025}
\end{equation}
Given the large spring stiffnesses for the link element, node 49 will have the same
displacements as node 50. Node 49 thus has the same $\hv{w}$ displacement
as node 50, while maintaining the required homogeneous form.

Most often, 6 link elements and corresponding dummy-node pairs are needed when all 
terms of $\bar \varepsilon_{ij}$ are non-zero.

\subsection{\texttt{DUMMY\_EPS\_MAP} definition}
The input file includes a \texttt{DUMMY\_EPS\_MAP} block with entries of the form:
\begin{center}
\texttt{i j dummy\_node driver\_node dof}
\end{center}
Interpretation of a map entry $(i,j)$:
\begin{itemize}
\item $(i,j)$ is the entry of the $\bar \varepsilon_{ij}$.
\item $\hv {dummy\_node}$ appears in the homogeneous MPC equation. Example: 
node 49 in Fig. 2.
\item $\hv{driver\_node}$ receives an absolute displacement constraint
on the indicated $\hv{dof}$  whose value equals $\bar{\varepsilon}_{ij}$.
Example: node 50 in Fig. 2.
\end{itemize}
The user connects $\hv{driver\_node}$ and $\hv{dummy\_node}$ using stiff link elements in the WARP3D model,
so that the imposed driver displacement is transmitted to the dummy DOF used in the MPCs.


\subsection{Resulting MPC structure}
For a generic paired-node relation, the script generates a WARP3D MPC equation of the form
\begin{equation}
u_i^{+}-u_i^{-} \;-\; \Delta x_1\,d_{i1} \;-\; \Delta x_2\,d_{i2} \;-\; \Delta x_3\,d_{i3} \;=\; 0,
\end{equation}
where $d_{ij}$ denotes the dummy-node DOF representing $\bar{\varepsilon}_{ij}$.
Terms are automatically omitted when:
\begin{itemize}
\item $\bar{\varepsilon}_{ij}=0$,
\item $\Delta x_j=0$, or
\item the referenced DOF is already fixed by an absolute constraint.
\end{itemize}

\subsection{Absolute constraints and closure iteration}
\subsubsection{Absolute constraints emitted by the script}
Absolute constraints are emitted under the WARP3D keyword \texttt{constraints}.
The script writes two classes of absolute constraints:
\begin{itemize}
\item User-specified zero constraints from the \texttt{ABS\_CONSTRAINTS} section of the input file.
\item Driver constraints implied by the strain tensor and the \texttt{DUMMY\_EPS\_MAP}:
\begin{equation}
u(\text{driver\_node},\text{dof}) = \bar{\varepsilon}_{ij}
\end{equation}
for each nonzero $\bar{\varepsilon}_{ij}$ present in the map.
\end{itemize}
A consistency check is performed: a dof cannot be both fixed to zero 
and assigned a nonzero driver value.

\subsubsection{Implied constraints and two-pass ``closure''}
When some strain components in $\bar{\varepsilon}_{ij}$ are zero 
and/or absolute constraints are 
applied, some MPC equations will be simplified.
If an MPC equation collapses to a single remaining \emph{physical} 
term, \eg $63\,u=0$, then
it becomes an implied absolute constraint rather than an MPC.
To detect these situations, the script performs an iterative ``closure'':
\begin{itemize}
\item generate MPCs under the current set of absolute constraints,
\item collect any implied absolute constraints,
\item add them to the absolute constraint set,
\item repeat until no new implied constraints are found,
thereby ensuring that all constraints are output in valid WARP3D order and that no incomplete MPCs remain.
\end{itemize}

\subsection{Output file structure and WARP3D ordering}
The output file generated by the script begins with a comment header describing:
\begin{itemize}
\item input file name,
\item detected bounds and dimensions,
\item declared vs.\ detected dimension comparison,
\item the imposed strain tensor,
\item the \texttt{DUMMY\_EPS\_MAP} summary, and
\item any warnings.
\end{itemize}
The output then contains:
\begin{itemize}
\item a \texttt{constraints} block (absolute constraints), followed by
\item a \texttt{multipoint} block (MPC equations).
\end{itemize}
This ordering matches the WARP3D requirement that absolute constraints appear before MPC definitions.


\section{Robustness and large meshes}

Node-pairing is performed in the script using simple coordinate-key 
hashing (rounded transverse coordinates),
which is efficient for large boundary node sets.
The total number of MPC equations scales with the number of paired boundary 
nodes and three displacement components.

The script has been exercised on large RVEs ($10^5$ nodes) and produces
tens of thousands of MPC equations with no manual editing required in typical workflows.

Execution time for the Python script is a few seconds for large models.

\section{Example Problem}
\nid This example demonstrates the concepts described here and develops
 an input file for the Python program
$\hv{rve\_mpc\_generator.py}$. See Fig. 3 for the setup. 
The analysis is linear-elastic for simplicity. Extension for nonlinear behavior does not alter
the constraints generated by the program.
%
\begin{figure}[h]
\begin{center}
\includegraphics[trim=0.50in 2.8in 0.4in 0.0in, clip=true,scale=0.7,angle=0]{Fig_3.pdf} 
\caption{{\small Fig. 3. Small example problem to illustrate setup and 
use of the $\hv{rve\_mpc\_generator.py}$ Python program.} 
\label{fig:macroscale}}
%
\end{center} 
\end{figure}

The model has 8, $\hv{l3disop}$ elements (8-node isoparametrics); 
4 $\hv{link2}$ elements;
27 nodes for the prism; 4 dummy-node pairs (8 total nodes);
and a linear-elastic material.

Figure 4 lists the node coordinates and element connectivities.

The prism dimensions are $L_x=1, L_y=2, L_z=4$ to emphasize
that imposed strains are multiplied by these sizes to obtain
corresponding imposed displacements. The macroscale strain tensor is

\begin{equation}
\bar{\boldsymbol{\varepsilon}} =
\begin{bmatrix}
\bar{\varepsilon}_{11} & \bar{\varepsilon}_{12} & \bar{\varepsilon}_{13} \\
\bar{\varepsilon}_{21} & \bar{\varepsilon}_{22} & \bar{\varepsilon}_{23} \\
\bar{\varepsilon}_{31} & \bar{\varepsilon}_{32} & \bar{\varepsilon}_{33}
\end{bmatrix}
=
\begin{bmatrix}
0.1 & 0.2 & 0.5 \\
0.2 & 0.0 & 0.3 \\
0.5 & 0.3 & 0.0
\end{bmatrix}
\end{equation}

\nid where values are chosen for convenience to make obvious the results are correct. 

One dummy-node pair and $\hv{link2}$ element for each non-zero strain component are defined
(here the macroscale shear strains are symmetric).
The (2,2) and (3,3) strain 
values could be non-zero with addition of 2 more $\hv{link2}$ elements and 2 more dummy-node pairs.

Coordinates for dummy nodes 28-35 are set at 10.0 10.0 10.0. The values are 
immaterial and do not affect the solution.
$\hv{link2}$ stiffnesses are set at $10^{10}$ -- preliminary analyses showed this value is
sufficient to make displacements at both nodes of $\hv{link2}$ elements identical.
%
\begin{figure}[h]
\begin{center}
\includegraphics[trim=0.0in 3.3in 0.60in 0.0in, clip=true,scale=0.7,angle=0]{Fig_4.pdf} 
\caption{{\small Fig. 4. Coordinates and incidences for the WARP3D input file.
Put into file named $\hv{coords\_incid.inp}$. }}
%
\end{center} 
\end{figure}  

The full WARP3D input file is shown in Fig. 5.  Figure 5a shows the included file of
constraints data.

\subsection{User generation of constraints}
\nid For this simple problem, it is feasible to generate the absolute and
MPC constraints manually. The next section uses the Python script.

\nid Recommended steps:
\begin{itemize}
\item Decide upon absolute constraints to prevent 3 rigid-body 
translations and 3 rotations (rigid-body constraints).
For this loading in Eq. (11), only node 1 needs to be constrained ($u=v=w=0$).
The non-zero imposed strains are sufficient to prevent the three rigid-rotations.

\item Start with all equations in Table 1. Eliminate all the zero terms 
from the zero imposed $\bar\varepsilon_{ij}$ values and constraints to remove
rigid-body motions.

\item Some of the MPCs to enforce the periodic boundary 
conditions may then have only 1 term remaining. The MPC has become 
an absolute constraint and must be included with them in the constraints.
In WARP3D input, absolute constraints may not be intermixed with MPCs.
\end{itemize}

\nid The imposed displacements for dummy-node loading become:

\begin{equation}
\begin{aligned}
\hv{Node\ 29:} &\quad u = 0.1,\ v = 0.0,\ w = 0.0\ \leftarrow \bar\epsilon_{11} \\
\hv{Node\ 31:} &\quad u = 0.2,\ v = 0.2,\ w = 0.0\ \leftarrow \bar\epsilon_{12}=\bar\epsilon_{21}\\
\hv{Node\ 33:} &\quad u = 0.5,\ v = 0.0,\ w = 0.5\ \leftarrow \bar\epsilon_{13}=\bar\epsilon_{31}\\ 
\hv{Node\ 35:} &\quad u = 0.0,\ v = 0.3,\ w = 0.3\ \leftarrow \bar\epsilon_{23}=\bar\epsilon_{32}
\end{aligned}
\end{equation} 
\begin{figure}[h]
\begin{center}
\includegraphics[trim=0.0in 2.8in 1.0in 0.0in, clip=true,scale=0.7,angle=0]{Fig_5.pdf} 
\caption{{\small Fig. 5. WARP3D input file for the example problem.}}
\end{center} 
\end{figure}

\begin{figure}[h]
\begin{center}
\includegraphics[trim=0.0in 2.8in 1.0in 0.0in, clip=true,scale=0.7,angle=0]{Fig_5a.pdf} 
\caption{{\small Fig. 5a. Manually prepared constraints input file.}}
\end{center} 
\end{figure}


\subsection{Constraints using $\hv{rve\_mpc\_generator.py}$}

\nid Figure 6 shows the input file prepared to generate all constraints
for the example problem. The required input for a very much larger
model differs only in the amount of nodal coordinates data.

At execution, the script requests the input file name, the file name for
generated constraints data and whether or not the symbolic form of all
the MPC equations should be printed as comment lines in the constraints
data file -- useful for understanding the conversion of all MPCs in
Table 1 for the model. 

Figure 7 shows the generated constraints file in form for direct
input to WARP3D. 
\begin{figure}[h]
\begin{center} 
\includegraphics[trim=0.0in 0.3in 1.0in 0.0in, clip=true,scale=0.7,angle=0]{Fig_6.pdf} 
\caption{{\small Fig. 6. Input file for the $\hv{rve\_mpc\_generator.py}$ Python program }}
\end{center} 
\end{figure}
\begin{figure}[h]
\begin{center} 
\includegraphics[trim=0.0in 1.6in 1.0in 0.0in, clip=true,scale=0.7,angle=0]{Fig_7.pdf} 
\caption{{\small Fig. 7. Output file of WARP3D constraints data generated by the script.}}
\end{center} 
\end{figure}


\end{document}

